% Kapitel 8
\section{Hydrostatik}

\subsection{Schweredruck}
Der Schweredruck ist nur von der Höhe der Flüssigkeit über dem Messpunkt abhängig.
Und nicht vom Volumen oder der Form des darüberliegenden Volumen.

$p_0$ bezeichnet den Normaldruck auf Meereshöhe. Dies entspritcht 1 Bar oder 100000 Pascal.

$\rho_0$ bezeichnet die Luftdichte auf Meereshöhe.

Der Schweredruck eines Gases (mit $h$ Höhe über Meer) 
\[
    p(h) := p_0 e^{-\frac{\rho_0}{p_0}gh}
\]

\subsection{Staitscher Auftrieb}
Jeder Festkörper in einer Flüssigkeit erfährt Auftrieb
\[
    F_A=\rho_\text{Flüssigkeit}V_\text{Körper}g
\]
Die Auftriebskraft ist eigentlich die differenz Kraft von der Flüssigkeit auf die untere
Fläche minus der Kraft von der Flüssigkeit, welche auf die obere Fläche wirkt.
Die Kräfte auf die Seitenflächen heben sich gegenseitig auf.
\begin{itemize}
\item[\underline{Archimedes:}] Auftrieb = Gewicht der verdrängten Flüssigkeit
\item[\underline{Stevin:}] Körper durch Flüssigkeitsvolumen ersetzt $\Rightarrow$ Gleichgewicht
\end{itemize}
% TODO



\subsection{Grenzflächen}
% TODO

\subsubsection{Oberflächenspannung}
% TODO

\subsubsection{Grenzflächenspannung}
% TODO

\subsubsection{Kapillarität}
% TODO
