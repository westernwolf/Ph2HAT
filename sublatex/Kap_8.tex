% Kapitel 8
\section{Hydrostatik}

\subsection{Schweredruck}
Der Schweredruck ist nur von der Höhe der Flüssigkeit über dem Messpunkt abhängig.
Und nicht vom Volumen oder der Form des darüberliegenden Volumen.

$p_0$ bezeichnet den Normaldruck auf Meereshöhe. Dies entspritcht 1 Bar oder 100000 Pascal.

$\rho_0$ bezeichnet die Luftdichte auf Meereshöhe.

Der Schweredruck eines Gases (mit $h$ Höhe über Meer) 
\[
    p(h) := p_0 e^{-\frac{\rho_0}{p_0}gh}
\]

\subsection{Staitscher Auftrieb}
Jeder Festkörper in einer Flüssigkeit erfährt Auftrieb
\[
    F_A=\rho_\text{Flüssigkeit}V_\text{Körper}g
\]
Die Auftriebskraft ist eigentlich die differenz Kraft von der Flüssigkeit auf die untere
Fläche minus der Kraft von der Flüssigkeit, welche auf die obere Fläche wirkt.
Die Kräfte auf die Seitenflächen heben sich gegenseitig auf.
\begin{itemize}
\item[\underline{Archimedes:}] Auftrieb = Gewicht der verdrängten Flüssigkeit
\item[\underline{Stevin:}] Körper durch Flüssigkeitsvolumen ersetzt $\Rightarrow$ Gleichgewicht
\end{itemize}



\subsection{Grenzflächen}

\subsubsection{Oberflächenspannung}
Oberflächenspannung = Kraft pro Länge

Zu wenige Moleküle an der Öberfläche (durch Verdunstung)
\[
    \sigma:=\frac{F}{l}
\]
Wenn man ein Draht-U aus der Flüssigkeit zieht, kann man die Höhe/Länge($l$) messen, wie wiet man die Flüssigkeit
\kom{hinaufziehen} kann.

Druck in einer Seifenblase
\[
    p = \frac{2\sigma}{r}
\]


\subsubsection{Grenzflächenspannung}
Die Grenzfläche zweier verschiedenen Zahlen. Durch unterschiedliche Anzahl und Anordnung der Moleküle.

Benetzend (>90 Grad) und nicht benetzend (<90 Grad, wie z.B. Quecksilber), dies ist der Winkel zwischen Flüssigkeit und anderem Material. 180 Grad bedeuted, dass es so benetzend ist, dass eine mono-molekulare
Schicht entsteht (wie z.B. ein Ölfilm).

\subsubsection{Kapillarität}
Beispiel ist die \kom{U-Form} von Wasser in einem Messbecher.

Anhebung bei benetzendem Fall (Wie Wasser im Becher) oder Absenkung bei nicht-benetzendem Fall.
\begin{align*}
    \sigma &= \text{Totale Grenflächenspannung}\\
    \rho &= \text{Dichte der Flüssigkeit}\\
    r &= \text{Radies der Kapillare}\\
    h&=\text{Kapillarität}\\
    h&=\frac{2\sigma}{\rho g r}
\end{align*}
% TODO
